The end points for the tidal tails were computed from the position angle ($\mathrm{PA}$) and length $l$ of the tails reported in \citet{Myeong2017}. Equatorial coordinates $(\alpha_i,\delta_i)$ for the end points were computed as:

\begin{eqnarray*}
\Delta\alpha_i & = & l_i\sin{(\mathrm{PA}_i)}/\cos{\delta_c} \\
\Delta\delta_i & = & l_i\cos{(\mathrm{PA}_i)}
\end{eqnarray*}

where $(\alpha_c,\delta_c)=(66\fdg1854,-21\fdg 1869)$ are the cluster's central coordinates, from the \citet{Harris1996} catalogue. 

We realised the track as a linear interpolation of the end points and cluster coordinates. This is a good approximation given the small extent of the tails (18' and 11') and their linear appearance in Fig.~1 of \citet{Myeong2017}. 
The authors do not estimate a distance gradient, we assume a mean heliocentric distance for the track of 80.8~kpc as cited by the authors from the \citet{Harris1996} compilation.
