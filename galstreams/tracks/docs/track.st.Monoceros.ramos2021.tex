The Monoceros proper motion tracks correspond to the median tracks in Fig. 5 of \citet{Ramos2021} (data provided by P. Ramos private communication). The celestial track corresponds to the smoothed splined that better represents the mean Galactic latitude of the HEALpix where these structures are detected, as a function of galactic longitude. 

Monoceros extends further towards $l>200^\circ$, but here we limit the tracks to the data provided in the blind identification conducted by \citet{Ramos2021}. The authors report a single mean distance of 10.6~kpc, which we adopt here for the full track.

A fair consensus seems to have been reached in the literature in that Monoceros is not a tidal stream formed by an accreted galaxy \citep[see review by][]{Yanny2016}, as originally thought \citep{Yanny2003}, but rather a feature excited or perturbed from the disc \citep{Kazantzidis2009,Laporte2019a,Laporte2019b}. In spite of this, we have chosen to keep it in the library given that its signature is localised in both the sky and proper motion spaces, and well represented by a simple track in each.
